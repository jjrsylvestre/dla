% *********************************************************************
% © 2023–2024 Jeremy Sylvestre
%
% Permission is granted to copy, distribute and/or modify this document
% under the terms of the GNU Free Documentation License, Version 1.3 or
% any later version published by the Free Software Foundation; with no
% Invariant Sections, no Front-Cover Texts, and no Back-Cover Texts. A
% copy of the license is included in the appendix entitled “GNU Free
% Documentation License” that appears in the output document of this
% PreTeXt source code. All trademarks™ are the registered® marks of
% their respective owners.
%
% *********************************************************************


% *********************************************************************
% operators
% *********************************************************************
\DeclareMathOperator{\RREF}{RREF}% row reduced echelon format
\DeclareMathOperator{\adj}{adj}% classical adjoint
\DeclareMathOperator{\proj}{proj}% projection
\DeclareMathOperator{\matrixring}{M}% $m\times n$ matrices
\DeclareMathOperator{\poly}{P}% polynomials
\DeclareMathOperator{\Span}{Span}
\DeclareMathOperator{\rank}{rank}
\DeclareMathOperator{\nullity}{nullity}

% extra operators for two-semester version
\DeclareMathOperator{\nullsp}{null}% null space of a matrix (could really just use ker, but want to use this before concept of kernel is introduced)
\DeclareMathOperator{\uppermatring}{U}% $m\times n$ upper triangular matrices
\DeclareMathOperator{\trace}{trace} % trace, more explicitly than just tr
\DeclareMathOperator{\dist}{dist}% distance
\DeclareMathOperator{\negop}{neg}% negative operator
\DeclareMathOperator{\Hom}{Hom}% hom space
\DeclareMathOperator{\im}{im}% image of a function
% *********************************************************************


% *********************************************************************
% numbers and number systems
% *********************************************************************
\newcommand{\R}{\mathbb{R}}% real numbers

% complex numbers for two-semester version
\newcommand{\C}{\mathbb{C}}% complex numbers
\newcommand{\ci}{\mathrm{i}}% the imaginary number (\ci instead of \i because of conflict with an existing \i in some package somewhere)
\newcommand{\cconj}[1]{\bar{#1}}% complex conjugate
\newcommand{\lcconj}[1]{\overline{#1}}% long complex conjugate
\newcommand{\cmodulus}[1]{\left\lvert #1 \right\rvert}% complex modulus
% *********************************************************************


% *********************************************************************
% general constructs
% *********************************************************************
\newcommand{\bbrac}[1]{\bigl(#1\bigr)}% slightly larger enclosing brackets
\newcommand{\Bbrac}[1]{\Bigl(#1\Bigr)}% even larger enclosing brackets
\newcommand{\irst}[1][1]{{#1}^{\mathrm{st}}}
\newcommand{\ond}[1][2]{{#1}^{\mathrm{nd}}}
\newcommand{\ird}[1][3]{{#1}^{\mathrm{rd}}}
\newcommand{\nth}[1][n]{{#1}^{\mathrm{th}}}
\newcommand{\leftrightlinesubstitute}{\scriptstyle \overline{\phantom{xxx}}}
\newcommand{\inv}[2][1]{{#2}^{-{#1}}}% inverse
\newcommand{\abs}[1]{\left\lvert #1 \right\rvert}% absolute value
\newcommand{\degree}[1]{{#1}^\circ} % makeshift degree symbol
\newcommand{\blank}{-}% dash for blank in maps

% system of equations
\newenvironment{sysofeqns}[1]
  {\left\{\begin{array}{#1}}
  {\end{array}\right.}

% *********************************************************************


% *********************************************************************
% general math
% *********************************************************************
% only needed in two-semester version
\newcommand{\iso}{\simeq}% isomorphic objects
% *********************************************************************


% *********************************************************************
% geometry
% *********************************************************************
\newcommand{\abray}[1]{\overrightarrow{#1}}% ray through points in arg, or directed line segment
\newcommand{\abctriangle}[1]{\triangle #1}% triangle through points in arg
% *********************************************************************


% *********************************************************************
% matrices and vectors
% *********************************************************************
% aligned matrix
\newenvironment{amatrix}[1]
  {\left[\begin{array}{#1}}
  {\end{array}\right]}

\newcommand{\mtrxvbar}{\mathord{|}}% a vertical bar to give a visual of extending a column in a matrix
\newcommand{\utrans}[1]{{#1}^{\mathrm{T}}}% undergrad-style transpose
\newcommand{\rowredarrow}{\xrightarrow[\text{reduce}]{\text{row}}}% row reduce arrow
\newcommand{\bidentmatfour}{\begin{bmatrix} 1 \amp 0 \amp 0 \amp 0 \\ 0 \amp 1 \amp 0 \amp 0 \\ 0 \amp 0 \amp 1 \amp 0
 \\ 0 \amp 0 \amp 0 \amp 1\end{bmatrix}}% four-by-four identity
\newcommand{\uvec}[1]{\mathbf{#1}}% undergrad boldface vector
\newcommand{\zerovec}{\uvec{0}}% zero vector
\newcommand{\bvec}[2]{#1\,\uvec{#2}}% undergrad vector with components
\newcommand{\ivec}[1]{\bvec{#1}{i}}% i vector with component
\newcommand{\jvec}[1]{\bvec{#1}{j}}% j vector with component
\newcommand{\kvec}[1]{\bvec{#1}{k}}% k vector with component
\newcommand{\injkvec}[3]{\ivec{#1} - \jvec{#2} + \kvec{#3}}% i - j + k vector with components
\newcommand{\norm}[1]{\left\lVert #1 \right\rVert}
\newcommand{\unorm}[1]{\norm{\uvec{#1}}}% norm with undergrad vector style
\newcommand{\dotprod}[2]{#1 \bigcdot #2}% dot product
\newcommand{\udotprod}[2]{\dotprod{\uvec{#1}}{\uvec{#2}}}% dot product with undergrad vector style
\newcommand{\crossprod}[2]{#1 \times #2}% cross product
\newcommand{\ucrossprod}[2]{\crossprod{\uvec{#1}}{\uvec{#2}}}% cross product with undergrad vector style
\newcommand{\uproj}[2]{\proj_{\uvec{#2}} \uvec{#1}}% projection with undergrad vector style
\newcommand{\adjoint}[1]{{#1}^\ast}% adjoint
\newcommand{\matrixOfplain}[2]{{\left[#1\right]}_{#2}}% matrix of, do not use basis font
\newcommand{\rmatrixOfplain}[2]{{\left(#1\right)}_{#2}}% matrix of, round brackets, does not use basis font
\newcommand{\rmatrixOf}[2]{\rmatrixOfplain{#1}{\basisfont{#2}}}% matrix of, round brackets
\newcommand{\matrixOf}[2]{\matrixOfplain{#1}{\basisfont{#2}}}% matrix of


% more matrices and vectors for two-semester version
\newcommand{\invmatrixOfplain}[2]{\inv{\left[#1\right]}_{#2}}% inverse matrix of, do not use basis font
\newcommand{\invrmatrixOfplain}[2]{\inv{\left(#1\right)}_{#2}}% inverse matrix of, round brackets, does not use basis font
\newcommand{\invmatrixOf}[2]{\invmatrixOfplain{#1}{\basisfont{#2}}}% inverse matrix of
\newcommand{\invrmatrixOf}[2]{\invrmatrixOfplain{#1}{\basisfont{#2}}}% inverse matrix of, round brackets
\newcommand{\stdmatrixOf}[1]{\left[#1\right]}% standard matrix of a linear transformation, no basis specified
\newcommand{\ucobmtrx}[2]{P_{\basisfont{#1} \to \basisfont{#2}}}% arrow style change of basis matrix
\newcommand{\uinvcobmtrx}[2]{\inv{P}_{\basisfont{#1} \to \basisfont{#2}}}% arrow style inverse change of basis matrix
\newcommand{\uadjcobmtrx}[2]{\adjoint{P}_{\basisfont{#1} \to \basisfont{#2}}}% arrow style adjoint change of basis matrix
\newcommand{\coordmapplain}[1]{C_{#1}}% coordinate map to $\R^n$ or $\C^n$
\newcommand{\coordmap}[1]{\coordmapplain{\basisfont{#1}}}% coordinate map to $\R^n$ or $\C^n$
\newcommand{\invcoordmapplain}[1]{\inv{C}_{#1}}% inverse coordinate map from $\R^n$ or $\C^n$
\newcommand{\invcoordmap}[1]{\invcoordmapplain{\basisfont{#1}}}% inverse coordinate map from $\R^n$ or $\C^n$
\newcommand{\similar}{\sim}
\newcommand{\inprod}[2]{\left\langle\, #1,\, #2 \,\right\rangle}% inner product
\newcommand{\uvecinprod}[2]{\inprod{\uvec{#1}}{\uvec{#2}}}% inner product with vector notation of arguments
\newcommand{\orthogcmp}[1]{{#1}^{\perp}}
\newcommand{\vecdual}[1]{{#1}^\ast}% vector space dual
\newcommand{\vecddual}[1]{{#1}^{\ast\ast}}% vector space double dua
% *********************************************************************

% *********************************************************************
% analysis
% *********************************************************************
% only needed in two-semester version
\newcommand{\dd}[2]{\frac{d{#1}}{d#2}}% Leibniz derivative
\newcommand{\ddx}[1][x]{\dd{}{#1}}% Leibniz differential operator
\newcommand{\ddt}[1][t]{\dd{}{#1}}% Leibniz differential operator
\newcommand{\dydx}{\dd{y}{x}}% Leibniz derivative, filled in
\newcommand{\dxdt}{\dd{x}{t}}% Leibniz derivative, filled in
\newcommand{\dydt}{\dd{y}{t}}% Leibniz derivative, filled in
\newcommand{\intspace}{\;}
\newcommand{\integral}[4]{\int^{#2}_{#1} #3 \intspace d{#4}}% spaced out integral
\newcommand{\funcdef}[3]{#1\colon #2\to #3}
% *********************************************************************
